\documentclass{beamer}

\input{settings.tex}


\title{Shortest Path Planning}
\subtitle{Computational Intelligence, Lecture 13}
\author{by Sergei Savin}
\centering
\date{\mydate}


\begin{document}
\maketitle


\begin{frame}{Content}

%\begin{itemize}
%\item Mixed Integer Linear Programming (MILP)
%\item Mixed Integer Quadratic Programming (MIQP)
%\item Example: Footstep planning
%\item Big-M method relaxation
%\item Example: switching control
%\item Homework
%\end{itemize}

\end{frame}



\begin{frame}{Shortest path on a graph}
	%\framesubtitle{General form}
	\begin{flushleft}
		
		If we want to plan a path on a 2D map, we can represent obstacle-free space regions as a nodes, and possible transitions between the obstacle-free space regions as graph edges. 
		
		\begin{figure}
			\centering
			\includegraphics[width=0.4\linewidth]{GraphPathPlanning}
			\caption{Path planning as graph search; \scriptsize{Credit: https://demonstrations.wolfram.com/ProbabilisticRoadmapMethod/}}
			\label{fig:graphpathplanning}
		\end{figure}
		
		
	\end{flushleft}
\end{frame}



\begin{frame}{Shortest path on a graph}
%\framesubtitle{General form}
\begin{flushleft}

Consider a directed graph (each edge has a direction assigned to it):

% TODO: \usepackage{graphicx} required
\begin{figure}
	\centering
	\includegraphics[width=0.5\linewidth]{DirectedGraph}
	\caption[]{Directed graph; \scriptsize{Credit: https://github.com/HQarroum/directed-graph}}
	\label{fig:directedgraph}
\end{figure}

How can we find a shortest path from a start node to a finish node on it?

\end{flushleft}
\end{frame}

\begin{frame}
	%\framesubtitle{How do we know the state?}
	\begin{flushleft}
		
		\centering{\Huge SPP as LP}
		
	\end{flushleft}
\end{frame}

\begin{frame}{Shortest path (1)}
%\framesubtitle{General form}
\begin{flushleft}

We assign index variable $x_i$ to $i$-th edge; each index variable is positive $x_i \geq 0$. 

\bigskip

If $x_i = 1$ the edge is part of the path. We assume that otherwise $x_i = 0$ (which will be enforced by the other constraints). Adding a cost $d_i$ associated with each edge (e.g. Euclidean distance) we get a linear cost $l(\bo{x})$:

\begin{equation}
	l(\bo{x}) = \bo{x}\T \bo{d}
\end{equation}

 
\end{flushleft}
\end{frame}



\begin{frame}{Shortest path (2)}
	%\framesubtitle{General form}
	\begin{flushleft}
		
		Since each edge connects one node (e.g. node $u$) to another (e.g. node $v$), we can label all index variables $x$ with superscripts, denoting nodes that they connect - $x^{u,v}$.
		
		\bigskip
		
		Our goal will be to count how many path segments enter and leave each node. For any normal node the number will be equal:
		
		\begin{equation}
			-\sum_{\forall i} x^{i,v} + \sum_{\forall j} x^{v,j} = 0
		\end{equation}
		
	
	\end{flushleft}
\end{frame}




\begin{frame}{Shortest path (3)}
	%\framesubtitle{General form}
	\begin{flushleft}
		
		
		We know that for the starting node $s$, there will only be one path segment leaving it:
		
		\begin{equation}
			-\sum_{\forall i} x^{i,s} + \sum_{\forall j} x^{s,j} = 1
		\end{equation}
		
		For the finishing node  $f$ we have only one path segment entering it:
		
		\begin{equation}
			-\sum_{\forall i} x^{i,f} + \sum_{\forall j} x^{f,j} = -1
		\end{equation}
		
	\end{flushleft}
\end{frame}



\begin{frame}{Shortest path (4)}
	%\framesubtitle{General form}
	\begin{flushleft}
		
		
		Together the problem becomes:
		
		\begin{equation} \label{LP}
			\begin{aligned}
				& \underset{\mathbf{x}}{\text{minimize}}
				& & \bo{x}\T \bo{d}, \\
				& \text{subject to}
				& & \begin{cases} 
					-\sum_{\forall i} x^{i,v} + \sum_{\forall j} x^{v,j} = 0, \ \ \ \forall v \\ 
					-\sum_{\forall i} x^{i,s} + \sum_{\forall j} x^{s,j} = 1,  \\
					-\sum_{\forall i} x^{i,f} + \sum_{\forall j} x^{f,j} = -1,\\
					\bo{x} \geq 0.
				\end{cases}
				%
			\end{aligned}
		\end{equation}
	
		And with that, the problem can be solved as an LP.
		
	\end{flushleft}
\end{frame}



\begin{frame}{SPP code (1)}
	%\framesubtitle{Code}
	\begin{flushleft}
		
		\begin{lstlisting}[language=Matlab]
n = 5; V = randn(n, 2);
% Connectivity:
C = [1, 2; %edge 1
1, 3; %edge 2
2, 3; %edge 3
2, 4; %edge 4
3, 5; %edge 5
4, 5];%edge 6
nc = size(C, 1);
d = zeros(nc, 1);  %cost - distance
for i = 1:nc
	d(i) = norm(V(C(i, 2), :) - V(C(i, 1), :));
end
\end{lstlisting}

		
	\end{flushleft}
\end{frame}



\begin{frame}{SPP code (2)}
	%\framesubtitle{Code}
	\begin{flushleft}
		
		\begin{lstlisting}[language=Matlab]
cvx_begin
variable x(nc, 1)
minimize( dot(d, x) )
subject to
x >= zeros(nc, 1);
x(1) + x(2) ==  1;%v 1
-x(5) - x(6) == -1;%v 5

-x(1) + x(3) + x(4) == 0;%v 2
-x(2) - x(3) + x(5) == 0;%v 3
-x(4) + x(6)        == 0;%v 4
cvx_end
\end{lstlisting}
		
	\end{flushleft}
\end{frame}


\begin{frame}
	%\framesubtitle{How do we know the state?}
	\begin{flushleft}
		
		\centering{\Huge SPP via A* algorithm}
		
	\end{flushleft}
\end{frame}


\begin{frame}{A star search}
	%\framesubtitle{General form}
	\begin{flushleft}
		
		Another popular shortest path planning method for graphs is \emph{A star} (\emph{A*}). Unlike the previous method it does not involve optimization, but it requires a \emph{heuristic}.
		
		\bigskip
		
		To study A* we once more consider a graph whose edges have cost associated with them.
		
		\bigskip
		
		Let $p$ be a node of the graph that the program found a path to. Point $p$ has a predecessor point $a(p)$ - the last node in the path towards $p$. Since each predecessor knows its predecessor, it means we can recursively reconstruct the path from the point $p$ to the start.
		
	\end{flushleft}
\end{frame}


\begin{frame}{A star search}
	%\framesubtitle{General form}
	\begin{flushleft}
		
		Finding a path from the start to the point $p$ we construct a sequence of edges that we need to travel through - $e_1$, $e_2$, ...,  $e_n$. Each of these edges has a cost associated with them -  $c_1$, $c_2$, ...,  $c_n$. So, the cost of reaching a node $p$ is $g(p) = \sum_{i=1}^{n} c_i$.
		
		\bigskip
		
		If we have a heuristic $h(p)$ that (while more or less accurate) always \emph{underestimates} the cost to reaching goal from the node $p$, we can use A* to choose the next node in the path. We choose the node that minimizes the following cost function:
		
		\begin{equation}
			p_{next} = \underset{p}{\text{argmin}} (g(p)+h(p))
		\end{equation} 
			
	\end{flushleft}
\end{frame}


\begin{frame}{A star search - implementation}
	%\framesubtitle{General form}
	\begin{flushleft}
		
		In practice, when we can compute $g(p)$ much simpler. Given a new node $p_{next}$ and its predecessor $p_a$, and the cost associated with the edge connecting them $c_a$, we can assign the value of $g(p_{next})$ as:
		
		\begin{equation}
			g(p_{next}) := g(p_a) + c_a
		\end{equation}
	
		Heuristic might be difficult to formulate in general, but as long as each node has coordinates on a plane associated with it, Euclidean distance provides a suitable heuristic.
		
	\end{flushleft}
\end{frame}



\begin{frame}{A star search - implementation}
	%\framesubtitle{General form}
	\begin{flushleft}
		
		A grid can easily be seen as a graph, where adjacency implies connection.
		
		\begin{figure}
			\centering
			\includegraphics[width=0.7\linewidth]{a_-search-algorithm-1}
			\caption{Example of a grid with obstacles. Credit: \bref{https://www.geeksforgeeks.org/a-search-algorithm/}{geeksforgeeks.org}}
			\label{fig:a-search-algorithm-1}
		\end{figure}
		
		
	\end{flushleft}
\end{frame}




\myqrframe

\end{document}
