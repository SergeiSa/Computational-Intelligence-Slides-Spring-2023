\documentclass{beamer}

 \pdfmapfile{+sansmathaccent.map}


\mode<presentation>
{
  \usetheme{Warsaw} % or try Darmstadt, Madrid, Warsaw, Rochester, CambridgeUS, ...
  \usecolortheme{crane} % or try seahorse, beaver, crane, wolverine, ...
  \usefonttheme{serif}  % or try serif, structurebold, ...
  \setbeamertemplate{navigation symbols}{}
  \setbeamertemplate{caption}[numbered]
} 

\renewcommand{\familydefault}{\rmdefault}


\hypersetup{
    colorlinks=true,
    linkcolor=blue,
    filecolor=blue,      
    urlcolor=blue,
}

\usepackage{amsmath}
\usepackage{mathtools}

\DeclareMathOperator*{\argmin}{arg\,min}

\usepackage{subcaption}

\usepackage{tikz}
\tikzset{every picture/.style={line width=0.75pt}} %set default line width to 0.75pt        
%%%%%%%%%%%%%%%%%%%%%%%%%%%%%%%%%%%%%%%%%%%%%%%%%%%


\usepackage{listings}
\usepackage{color}
\definecolor{mygreen}{rgb}{0,0.6,0}
\definecolor{mygray}{rgb}{0.5,0.5,0.5}
\definecolor{mymauve}{rgb}{0.58,0,0.82}

\definecolor{cof}{RGB}{255,187,0}
\definecolor{pur}{RGB}{205,137,0}
\definecolor{greeo}{RGB}{91,173,69}
\definecolor{greet}{RGB}{52,111,72}

\usetikzlibrary{fadings}
\usetikzlibrary{patterns}
\usetikzlibrary{shadows.blur}
\usetikzlibrary{shapes}

\lstset{ 
  backgroundcolor=\color{white},   % choose the background color; you must add \usepackage{color} or \usepackage{xcolor}; should come as last argument
  basicstyle=\footnotesize,        % the size of the fonts that are used for the code
  breakatwhitespace=false,         % sets if automatic breaks should only happen at whitespace
  breaklines=true,                 % sets automatic line breaking
  captionpos=b,                    % sets the caption-position to bottom
  commentstyle=\color{mygreen},    % comment style
  deletekeywords={...},            % if you want to delete keywords from the given language
  escapeinside={\%*}{*)},          % if you want to add LaTeX within your code
  extendedchars=true,              % lets you use non-ASCII characters; for 8-bits encodings only, does not work with UTF-8
  firstnumber=0000,                % start line enumeration with line 0000
  frame=single,	                   % adds a frame around the code
  keepspaces=true,                 % keeps spaces in text, useful for keeping indentation of code (possibly needs columns=flexible)
  keywordstyle=\color{blue},       % keyword style
  language=Octave,                 % the language of the code
  morekeywords={*,...},            % if you want to add more keywords to the set
  numbers=left,                    % where to put the line-numbers; possible values are (none, left, right)
  numbersep=5pt,                   % how far the line-numbers are from the code
  numberstyle=\tiny\color{mygray}, % the style that is used for the line-numbers
  rulecolor=\color{black},         % if not set, the frame-color may be changed on line-breaks within not-black text (e.g. comments (green here))
  showspaces=false,                % show spaces everywhere adding particular underscores; it overrides 'showstringspaces'
  showstringspaces=false,          % underline spaces within strings only
  showtabs=false,                  % show tabs within strings adding particular underscores
  stepnumber=2,                    % the step between two line-numbers. If it's 1, each line will be numbered
  stringstyle=\color{mymauve},     % string literal style
  tabsize=2,	                   % sets default tabsize to 2 spaces
  title=\lstname                   % show the filename of files included with \lstinputlisting; also try caption instead of title
}




%%%%%%%%%%%%%%%%%%%%%%%%%%%%%%%%%%%%%%%%%%%%%%%%%%%%%


\title{ Dynamical systems with constraints }
\subtitle{Computational Intelligence, Lecture 12}
\author{by Sergei Savin}
\centering
\date{Fall 2020}



\begin{document}
\maketitle


\begin{frame}{Content}

\begin{itemize}
\item Mechanical systems with constraints
\begin{itemize}
    \item Lagrange equations
    \item Manipulator equations
    \item Full description of constrained dynamics
    \item Linear description of constrained dynamics
\end{itemize}
\item Implicit (minimal) representation of a constrained system
\item Inverse dynamics
\begin{itemize}
    \item LTI
    \item Manipulator equations
    \item Quadratic program
\end{itemize}
\item Homework
\end{itemize}

\end{frame}



\begin{frame}{Mechanical systems with constraints}
\framesubtitle{Lagrange equations}
\begin{flushleft}

\emph{Lagrange equations} are a basic tool for modelling multi-body mechanical systems. General form is given as:

\begin{equation}
    \frac{d}{dt} \frac{\partial T}{\partial \dot{\mathbf{q}}} - \frac{\partial T}{\partial \mathbf{q}} = \tau
\end{equation}

where $T$ is kinetic energy of the system, $\mathbf{q}$ is the vector of generalized coordinates and $\tau$ is the vector of generalized torques.
 
\bigskip

For systems with \emph{constraints}, for example $\mathbf{f}(\mathbf{q}) = 0$, Lagrange equations take a different form:

\begin{equation}
    \frac{d}{dt} \frac{\partial T}{\partial \dot{\mathbf{q}}} - \frac{\partial T}{\partial \mathbf{q}} = \tau + \frac{\partial \mathbf{f}}{\partial \mathbf{q}}^\top \lambda
\end{equation}
 
\end{flushleft}
\end{frame}




\begin{frame}{Mechanical systems with constraints}
\framesubtitle{Manipulator equations}
\begin{flushleft}

Lagrange equations can be easily converted to a \emph{manipulator equations} form:

\begin{equation}
\label{eq:manipulator}
    \mathbf{H}\ddot{\mathbf{q}} + \mathbf{C}\dot{\mathbf{q}} + \mathbf{g} = \mathbf{T}\mathbf{u}
\end{equation}

where $\mathbf{H}$ is a generalized inertia matrix, $\mathbf{C}$ is a matrix of generalized inertial forces, $\mathbf{g}$ is a generalized gravity force, $\mathbf{T}$ is a control map (control matrix), and $\mathbf{u}$ is a vector of control inputs.
 
\bigskip

Manipulator equations for systems with constraints $\mathbf{f}(\mathbf{q}) = 0$ are very similar:
 
\begin{equation}
\label{eq:manipulator:constrained}
    \mathbf{H}\ddot{\mathbf{q}} + \mathbf{C}\dot{\mathbf{q}} + \mathbf{g} = \mathbf{T}\mathbf{u} + \mathbf{F}^\top \lambda
\end{equation}

where $\mathbf{F} = \frac{\partial \mathbf{f}}{\partial \mathbf{q}}$ and $\lambda$ are Lagrange multipliers, encoding reaction forces ta ht enforce constraints.

\end{flushleft}
\end{frame}



\begin{frame}{Mechanical systems with constraints}
\framesubtitle{Full description of constrained dynamics}
\begin{flushleft}

It is not sufficient to provide \eqref{eq:manipulator:constrained} to describe the behaviour of the system, since it depends both of the differential equation and the algebraic constraint.

\bigskip

If we want to study how generalized accelerations $\ddot{\mathbf{q}}$ behave, we can differentiate $\mathbf{f}(\mathbf{q}) = 0$ twice and add the result to the manipulator equation:
 
\begin{equation}
\begin{cases}
\label{eq:manipulator:constrained2}
    \mathbf{H}\ddot{\mathbf{q}} + \mathbf{C}\dot{\mathbf{q}} + \mathbf{g} = \mathbf{T}\mathbf{u} + \mathbf{F}^\top \lambda \\
    \mathbf{F}\ddot{\mathbf{q}} + \dot{\mathbf{F}}\dot{\mathbf{q}} = 0
\end{cases}
\end{equation}

Equation \eqref{eq:manipulator:constrained2} can be solved with respect to variables $\ddot{\mathbf{q}}$ and $\lambda$.

\end{flushleft}
\end{frame}



\begin{frame}{Mechanical systems with constraints}
\framesubtitle{Linear description of constrained dynamics}
\begin{flushleft}

Introducing change of variables $\mathbf{x} = \begin{bmatrix} \mathbf{q} \\ \dot{\mathbf{q}} \end{bmatrix}$ we can linearize dynamics \eqref{eq:manipulator} to obtain form:

\begin{equation}
\label{eq:LTI}
    \dot{\mathbf{x}} = \mathbf{A} \mathbf{x} + \mathbf{B} \mathbf{u} + \mathbf{c}
\end{equation}

where $\mathbf{A}$ is the state matrix, $\mathbf{B}$ is the control matrix and $\mathbf{c}$ is the affine term of the affine dynamics model.

\bigskip

For systems with constraints the same linearization takes form:

\begin{equation}
\label{eq:CLTI}
\begin{cases}
    \dot{\mathbf{x}} = \mathbf{A} \mathbf{x} + \mathbf{B} \mathbf{u} + \mathbf{S} \lambda + \mathbf{c} \\
    \mathbf{G}\dot{\mathbf{x}} = 0
\end{cases}    
\end{equation}

where $\mathbf{S}$ is linearized constraint Jacobian and $\mathbf{G} = \begin{bmatrix} \mathbf{F} & \mathbf{0} \\ \dot{\mathbf{F}} & \mathbf{F} \end{bmatrix}$. 

\end{flushleft}
\end{frame}



\begin{frame}{Implicit (minimal) representation of a constrained system}
\framesubtitle{Part 1}
\begin{flushleft}

We can observe that constraint $\mathbf{G}\dot{\mathbf{x}} = 0$ implies that all feasible state velocities $\dot{\mathbf{x}}$ lie in the null space of $\mathbf{G}$. This means that we can introduce a new lower dimensional variable $\mathbf{z}$ to describe $\mathbf{x}$ (assuming initial value of $\mathbf{x}$ lies in the column space of $\mathbf{N}$):

\begin{equation}
    \mathbf{N}\mathbf{z} = \mathbf{x}
\end{equation}

where $\mathbf{N} = \text{null}(\mathbf{G})$ - orthonormal basis in the null space of $\mathbf{G}$. 
\end{flushleft}
\end{frame}


\begin{frame}{Implicit (minimal) representation of a constrained system}
\framesubtitle{Part 2}
\begin{flushleft}

Let us re-express dynamics \eqref{eq:CLTI} in terms of $\mathbf{z}$ by multiplying it by $\mathbf{N}^\top$ on the left:

\begin{equation}
    \mathbf{N}^\top \dot{\mathbf{x}} = \mathbf{N}^\top \mathbf{A} \mathbf{x} + \mathbf{N}^\top \mathbf{B} \mathbf{u} + \mathbf{N}^\top \mathbf{S} \lambda + \mathbf{N}^\top \mathbf{c}
\end{equation}

We can prove that $\mathbf{N}^\top \mathbf{S} = 0$ for all mechanical systems (for example, by observing that mechanical constrains do not do work) or check that our particular $\mathbf{S}$ lies in the row space of our $\mathbf{G}$.

Noting that $\dot{\mathbf{z}} = \mathbf{N}^\top \dot{\mathbf{x}}$ and $\mathbf{x} = \mathbf{N}\mathbf{z}$ we get:

\begin{equation}
    \dot{\mathbf{z}} = \mathbf{N}^\top \mathbf{A} \mathbf{N} \mathbf{z} + \mathbf{N}^\top \mathbf{B} \mathbf{u} + \mathbf{N}^\top \mathbf{c}
\end{equation}

Defining $\mathbf{A}_N = \mathbf{N}^\top \mathbf{A} \mathbf{N}$, $\mathbf{B}_N = \mathbf{N}^\top \mathbf{B}$ and $\mathbf{c}_N = \mathbf{N}^\top \mathbf{c}$ we get:

\begin{equation}
    \dot{\mathbf{z}} = \mathbf{A}_N \mathbf{z} + \mathbf{B}_N \mathbf{u} + \mathbf{c}_N
\end{equation}

\end{flushleft}
\end{frame}



\begin{frame}{Implicit (minimal) representation of a constrained system}
\framesubtitle{Part 3}
\begin{flushleft}

Since we achieved that our constrained dynamics is written in the standard LTI form:

\begin{equation}
    \dot{\mathbf{z}} = \mathbf{A}_N \mathbf{z} + \mathbf{B}_N \mathbf{u} + \mathbf{c}_N,
\end{equation}

we can use standard LTI control methods on it, for example finding optimal feedback gains via pole placement or LQR:

\begin{equation}
    \mathbf{K}_N = \text{lqr}(\mathbf{A}_N, \mathbf{B}_N, \mathbf{Q}, \mathbf{R})
\end{equation}

where $\mathbf{Q}$ and $\mathbf{R}$ are matrices defining cost function for the LQR problem.

\end{flushleft}
\end{frame}



\begin{frame}{Inverse dynamics}
\framesubtitle{LTI}
\begin{flushleft}

For any LTI system, including the LTI form of a constrained system we saw previously, inverse dynamics can be solved precisely by a pseudo-inverse, as long as there exist a solution. The following condition verifies it:

\begin{equation}
    (\mathbf{I} - \mathbf{B}\mathbf{B}^+)(\dot{\mathbf{x}} - \mathbf{A} \mathbf{x} - \mathbf{c}) = 0,
\end{equation}

The condition checks if vector $(\dot{\mathbf{x}} - \mathbf{A} \mathbf{x} - \mathbf{c})$ lies in the column space of $\mathbf{B}$. If it holds, precise solution to inverse kinematics can be found as:

\begin{equation}
    \mathbf{u}_{ID} = \mathbf{B}^+(\dot{\mathbf{x}} - \mathbf{A} \mathbf{x} - \mathbf{c}).
\end{equation}

\end{flushleft}
\end{frame}



\begin{frame}{Inverse dynamics}
\framesubtitle{Manipulator equations}
\begin{flushleft}

For a constrained mechanical system we can solve inverse dynamics without the need for linearization. Consider the following dynamics:

\begin{equation}
    \mathbf{H}\ddot{\mathbf{q}} + \mathbf{C}\dot{\mathbf{q}} + \mathbf{g} = \mathbf{T}\mathbf{u} + \mathbf{F}^\top \lambda
\end{equation}

For that we represent constraint Jacobian $\mathbf{F}^\top$ as its QR decomposition: $\mathbf{F}^\top = \mathbf{Q} \begin{bmatrix} \mathbf{R} \\ \mathbf{0}  \end{bmatrix}$, where $\mathbf{Q}^\top \mathbf{Q} = \mathbf{Q} \mathbf{Q}^\top = \mathbf{I}$ and $\mathbf{R}$ is invertible.

\begin{equation}
    \mathbf{H}\ddot{\mathbf{q}} + \mathbf{C}\dot{\mathbf{q}} + \mathbf{g} = \mathbf{T}\mathbf{u} + \mathbf{Q} \begin{bmatrix} \mathbf{R} \\ \mathbf{0}  \end{bmatrix} \lambda
\end{equation}


\end{flushleft}
\end{frame}


\begin{frame}{Inverse dynamics}
\framesubtitle{Manipulator equations, part 2}
\begin{flushleft}

Let us multiply the equation by $\mathbf{Q}^\top$:

\begin{equation}
    \mathbf{Q}^\top (\mathbf{H}\ddot{\mathbf{q}} + \mathbf{C}\dot{\mathbf{q}} + \mathbf{g}) = \mathbf{Q}^\top\mathbf{T}\mathbf{u} + \begin{bmatrix} \mathbf{R} \\ \mathbf{0}  \end{bmatrix} \lambda
\end{equation}

Introducing switching variables (to divide upper and lower part of the equations) $\mathbf{S}_1 = \begin{bmatrix} \mathbf{I} & \mathbf{0}  \end{bmatrix}$ and $\mathbf{S}_2 = \begin{bmatrix} \mathbf{0} & \mathbf{I}  \end{bmatrix}$ and multiplying equations by one and the other we get two systems:

\begin{equation}
\begin{cases}
    \mathbf{S}_1 \mathbf{Q}^\top (\mathbf{H}\ddot{\mathbf{q}} + \mathbf{C}\dot{\mathbf{q}} + \mathbf{g}) = \mathbf{S}_1\mathbf{Q}^\top\mathbf{T}\mathbf{u} + \mathbf{R} \lambda \\
    \mathbf{S}_2 \mathbf{Q}^\top (\mathbf{H}\ddot{\mathbf{q}} + \mathbf{C}\dot{\mathbf{q}} + \mathbf{g}) = \mathbf{S}_2\mathbf{Q}^\top\mathbf{T}\mathbf{u}
\end{cases}
\end{equation}

The main advantage we achieved is that now we can calculate both $\mathbf{u}$ and $\lambda$

\end{flushleft}
\end{frame}


\begin{frame}{Inverse dynamics}
\framesubtitle{Manipulator equations, part 3}
\begin{flushleft}

Resulting expression for $\mathbf{u}$ is:

\begin{equation}
    \mathbf{u} = 
    (\mathbf{S}_2\mathbf{Q}^\top\mathbf{T})^+ \mathbf{S}_2 \mathbf{Q}^\top (\mathbf{H}\ddot{\mathbf{q}} + \mathbf{C}\dot{\mathbf{q}} + \mathbf{g})
\end{equation}

Expression for $\lambda$ is:

\begin{equation}
\lambda = \mathbf{R}^{-1} \mathbf{S}_1 \mathbf{Q}^\top (\mathbf{H}\ddot{\mathbf{q}} + \mathbf{C}\dot{\mathbf{q}} + \mathbf{g} - \mathbf{T}\mathbf{u})
\end{equation}

We can notice a pseudo-inverse, implying that the no-residual solution does not have to exist.

\end{flushleft}
\end{frame}



\begin{frame}{Inverse dynamics}
\framesubtitle{Quadratic program}
\begin{flushleft}

We can easily write inverse dynamics as a QP:


\begin{equation}
\begin{aligned}
& \underset{\mathbf{u}, \lambda}{\text{minimize}}
& & ||\mathbf{u}||, \\
& \text{subject to}
& & \begin{cases}
    \mathbf{H}\ddot{\mathbf{q}} + \mathbf{C}\dot{\mathbf{q}} + \mathbf{g} = \mathbf{T}\mathbf{u} + \mathbf{F}^\top \lambda \\
    \mathbf{F}\ddot{\mathbf{q}} + \dot{\mathbf{F}}\dot{\mathbf{q}} = 0
    \end{cases}
\end{aligned}
\end{equation}

If there are some constraints or limits on the control input (torque limits, for instance) or the reaction forces are restricted (by friction cones, for instance), those can be directly added.

\end{flushleft}
\end{frame}


\begin{frame}{Homework}
% \framesubtitle{Parameter estimation}
\begin{flushleft}

Implement a model-predictive controller for an LTI system with implicit or explicit constraints.

\end{flushleft}
\end{frame}



\begin{frame}
\centerline{Lecture slides are available via Moodle.}
\bigskip
\centerline{You can help improve these slides at:}

\centerline{\href{https://github.com/SergeiSa/Computational-Intelligence-Slides-Fall-2020}{github.com/SergeiSa/Computational-Intelligence-Slides-Fall-2020}}


\bigskip
\centerline{Check Moodle for additional links, videos, textbook suggestions.}
\end{frame}

\end{document}
