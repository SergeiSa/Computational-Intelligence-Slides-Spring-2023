\documentclass{beamer}

\input{settings.tex}


\title{Path planning}
\subtitle{Computational Intelligence, Lecture ?}
\author{by Sergei Savin}
\centering
\date{\mydate}


\begin{document}
\maketitle


\begin{frame}{Content}

%\begin{itemize}
%\item Mixed Integer Linear Programming (MILP)
%\item Mixed Integer Quadratic Programming (MIQP)
%\item Example: Footstep planning
%\item Big-M method relaxation
%\item Example: switching control
%\item Homework
%\end{itemize}

\end{frame}



\begin{frame}{Shortest path on a graph}
	%\framesubtitle{General form}
	\begin{flushleft}
		
		If we want to plan a path on a 2D map, we can represent obstacle-free space regions as a nodes, and possible transitions between the obstacle-free space regions as graph edges. 
		
		\begin{figure}
			\centering
			\includegraphics[width=0.4\linewidth]{GraphPathPlanning}
			\caption{Path planning as graph search; \scriptsize{Credit: https://demonstrations.wolfram.com/ProbabilisticRoadmapMethod/}}
			\label{fig:graphpathplanning}
		\end{figure}
		
		
	\end{flushleft}
\end{frame}



\begin{frame}{Shortest path on a graph}
%\framesubtitle{General form}
\begin{flushleft}

Consider a directed graph (each edge has a direction assigned to it):

% TODO: \usepackage{graphicx} required
\begin{figure}
	\centering
	\includegraphics[width=0.5\linewidth]{DirectedGraph}
	\caption[]{Directed graph; \scriptsize{Credit: https://github.com/HQarroum/directed-graph}}
	\label{fig:directedgraph}
\end{figure}

How can we find a shortest path from a start node to a finish node on it?

\end{flushleft}
\end{frame}



\begin{frame}{Shortest path (1)}
%\framesubtitle{General form}
\begin{flushleft}

We assign index variable $x_i$ to $i$-th edge; each index variable is positive $x_i \geq 0$. 

\bigskip

If $x_i = 1$ the edge is part of the path. We assume that otherwise $x_i = 0$ (which will be enforced by the other constraints). Adding a cost $d_i$ associated with each edge (e.g. Euclidean distance) we get a linear cost $l(\bo{x})$:

\begin{equation}
	l(\bo{x}) = \bo{x}\T \bo{d}
\end{equation}

 
\end{flushleft}
\end{frame}



\begin{frame}{Shortest path (2)}
	%\framesubtitle{General form}
	\begin{flushleft}
		
		Since each edge connects one node (e.g. node $u$) to another (e.g. node $v$), we can label all index variables $x$ with superscripts, denoting nodes that they connect - $x^{u,v}$.
		
		\bigskip
		
		Our goal will be to count how many path segments enter and leave each node. For any normal node the number will be equal:
		
		\begin{equation}
			-\sum_{\forall i} x^{i,v} + \sum_{\forall j} x^{v,j} = 0
		\end{equation}
		
	
	\end{flushleft}
\end{frame}




\begin{frame}{Shortest path (3)}
	%\framesubtitle{General form}
	\begin{flushleft}
		
		
		We know that for the starting node $s$, there will only be one path segment leaving it:
		
		\begin{equation}
			-\sum_{\forall i} x^{i,s} + \sum_{\forall j} x^{s,j} = 1
		\end{equation}
		
		For the finishing node  $f$ we have only one path segment entering it:
		
		\begin{equation}
			-\sum_{\forall i} x^{i,f} + \sum_{\forall j} x^{f,j} = -1
		\end{equation}
		
	\end{flushleft}
\end{frame}



\begin{frame}{Shortest path (4)}
	%\framesubtitle{General form}
	\begin{flushleft}
		
		
		Together the problem becomes:
		
		\begin{equation} \label{LP}
			\begin{aligned}
				& \underset{\mathbf{x}}{\text{minimize}}
				& & \bo{x}\T \bo{d}, \\
				& \text{subject to}
				& & \begin{cases} 
					-\sum_{\forall i} x^{i,v} + \sum_{\forall j} x^{v,j} = 0, \ \ \ \forall v \\ 
					-\sum_{\forall i} x^{i,s} + \sum_{\forall j} x^{s,j} = 1,  \\
					-\sum_{\forall i} x^{i,f} + \sum_{\forall j} x^{f,j} = -1,\\
					\bo{x} \geq 0.
				\end{cases}
				%
			\end{aligned}
		\end{equation}
	
		And with that, the problem can be solved as an LP.
		
	\end{flushleft}
\end{frame}



\myqrframe

\end{document}
